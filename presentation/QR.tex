\documentclass[serbian]{beamer}

\usepackage{babel}
\mode<presentation>
{
  \usetheme{Warsaw}      % or try Boadilla, Rochester, ...
  \usecolortheme{beaver} % or try beaver, dove, ...
  \usefonttheme{default}  % or try serif, structurebold, ...
  \setbeamertemplate{navigation symbols}{}
  \setbeamertemplate{caption}[numbered]
  \setbeamercolor{item}{fg=red}
} 

\usepackage[style=numeric]{biblatex}
\addbibresource{QR.bib}
\setbeamertemplate{bibliography item}[triangle]

% This is for slide numbering
\setbeamertemplate{sidebar right}{}
\setbeamertemplate{footline}{%
	\hfill\usebeamertemplate***{navigation symbols}
	\large\insertframenumber{}/\inserttotalframenumber\hspace{0.5cm}\vspace{0.2cm}
}

\usepackage{listings}
\lstset{
	basicstyle=\ttfamily\small,
	columns=fullflexible,
	keepspaces=true,
	showstringspaces=false,
	escapechar=\%,
	tabsize=4
}

\usepackage{amsmath}
\newcommand\numberthis{\addtocounter{equation}{1}\tag{\theequation}}

\title[QR dekompozicija]{QR dekompozicija}
\author{Anja Bukurov}
\institute{Matematički fakultet, \\ Univerzitet u Beogradu}
\date{\today.}

\begin{document}

\begin{frame}
  	\titlepage
\end{frame}

\begin{frame}{Sadržaj}
  	\tableofcontents
\end{frame}

\section{Uvod}

\begin{frame}{Uvod}

	Neka je $A$ matrica dimenzija $m\times n$ ($m > n$) i ranga $n$.
	\\ ~
	\\
	Matrica $A$ može se razložiti na proizvod
	$$A = QR$$
	gde je $Q$ ortogonalna matrica dimenzija $m \times n$, a $R$ je gornjetrougaouna matrica dimenzija $n \times n$.
	
\end{frame}

\section{Algoritmi}

\subsection{Klasičan Šmitov algoritam}
\begin{frame}{Klasičan Šmitov algoritam}

\begin{itemize}
	\item Algoritam se sastoji iz $n$ koraka
	\item U svakom koraku izračunata je jedna kolona matrice $Q$
	\item Posmatrajmo $k$-ti korak:
	\begin{itemize}
		\item Do sad su izračunate kolone $cQ_1 \ldots cQ_{k-1}$
		\item Prvo računamo $k$-tu kolonu matrice $R$ 
		$$R_{ik} = cQ_i^TcA_k$$
		\item Formiramo vektor $b_k = cA_k − \sum_{i=0}^{k-1}R_{ik}cQ_i$ koji je ortognalan na prethodno izračunate kolone $cQ_1 \ldots cQ_{k-1}$
		\item Ostaje još da se vektor normira i dobili smo $k$-tu kolonu.
	\end{itemize}
\end{itemize}

\end{frame}


\begin{frame}{Klasičan Šmitov algoritam}
	
	\begin{itemize}
		\item Matrica $Q$, koja se dobija ovim postupkom, nije ortogonalna i zbog toga je nestabilan
		\item Međutim, proizvod matrica $Q$ i $R$ savršeno se poklapa sa polaznom matricom $A$
		\item 
	\end{itemize}
	
\end{frame}

\subsection{Modifikacija Šmitovog algoritma}
\begin{frame}{Modifikacija Šmitovog algoritma}

	\begin{itemize}
		\item Modifikacija se oslanja na vezu $A = QR$ u malo drugačijem obliku
		\item Zapisujemo proizvod kao sumu matrica ranga 1:
		
		$$A = \sum_{i=1}^{n} cQ_irR_i^T $$
		
		\item Pre nego što izračunamo matricu $Q$, računamo vrednosti matrice $R$ na glavnoj dijagonali kao normu $k$-te kolone matrice $A$	
	\end{itemize}

\end{frame}

\begin{frame}{Modifikacija Šmitovog algoritma}

	\begin{itemize}
		\item $k$-tu kolonu matrice $Q$ dobijamo normiranjem $k$-te kolone matrice $A$ odnosno deljenjem elemenata $k$-te kolone $k$-tim elementom dijagonale matrice $R$
		\item Konačnu vrednost $k$-te kolone dobijamo formulom
		$$cQ_k = cQ_{k-1} - cQ_{k-1}rR_k^T$$	
	\end{itemize}
\end{frame}

\subsection{Reortogonalizacija}
\begin{frame}{Reortogonalizacija}

\begin{itemize}
	\item Problem klasičnog Šmitovog algoritma jeste u računanju vektora $b$
	\item Ukoliko je norma vektora $b$ mnogo manja od norme $k$-te kolone matrice $A$ doći će do poništavanja
	
	\item Ideja je da se polazna matrica $A$ ortogonalizuje sve dok norme matrica iz susednih koraka ne postanu dovoljno bliske
\end{itemize}

\end{frame}



\begin{frame}

\vspace*{3em}

{\Large Hvala na pa\v znji!} 

\vspace*{1em}

Pitanja?

\end{frame}

\section{Dodatak}

\begin{frame}{Dodatak}

\begin{flushleft}
	{\Large Literatura}
\end{flushleft}

\printbibliography

\begin{flushleft}
	{\Large Repozitorijum}
\end{flushleft}

\href{https://github.com/djinx/QR-Decomposition}{https://github.com/djinx/QR-Decomposition}

\end{frame}

\end{document}